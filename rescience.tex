\documentclass[runningheads]{llncs}
%
\usepackage{graphicx}

% If you use the hyperref package, please uncomment the following line
% to display URLs in blue roman font according to Springer's eBook style:
% \renewcommand\UrlFont{\color{blue}\rmfamily}

\begin{document}
%
\title{ReScience~C: a journal for reproducible replications in computational science}
%
%\titlerunning{Abbreviated paper title}
% If the paper title is too long for the running head, you can set
% an abbreviated paper title here

\author{Nicolas P. Rougier\inst{1,2,3}\orcidID{0000-0002-6972-589X} \and
Konrad Hinsen\inst{4,5}\orcidID{0000-0003-0330-9428}}
%
\authorrunning{N.P. Rougier and K. Hinsen}
%
\institute{INRIA Bordeaux Sud-Ouest Talence, France
\email{Nicolas.Rougier@inria.fr}
\and
Institut des Maladies Neurod\'{e}g\'{e}n\'{e}ratives,
Universit\'{e} de Bordeaux, CNRS UMR 5293, Bordeaux, France
\and
LaBRI, Universit\'{e} de Bordeaux, Bordeaux INP, CNRS UMR 5800, Talence, France
\and
Centre de Biophysique Mol\'{e}culaire, CNRS UPR4301, Orléans, France 
\email{Konrad.Hinsen@cnrs.fr}
\and
Synchrotron SOLEIL, Division Exp\'{e}riences, Gif sur Yvette, France}
%
\maketitle              % typeset the header of the contribution
%
\begin{abstract}
  Independent replication is one of the most powerful methods to
  detect mistakes and inaccuracies in published scientific studies.
  In computational science, it requires the reimplementation of the
  methods described in the original article by a different team of
  researchers.  Replication is often performed by scientists who wish
  to gain a better understanding of a published method, but its
  results are rarely made public. ReScience~C is a peer-reviewed
  journal dedicated to the publication of high-quality computational
  replications that provide added value to the scientific community.
  To this end, ReScience~C requires replications to be reproducible
  and implemented using Open Source languages and libraries. In this
  article, we provide an overview of ReScience~C's goals and quality
  standard, outline the submission and reviewing processes, and
  summarize the experience of its first three years of operation,
  concluding with an outlook towards evolutions planned for the near
  future.

  \keywords{Open science \and Computational science \and Reproducibility}
\end{abstract}
%
\section{Introduction}

%
% ---- Bibliography ----
%
\bibliographystyle{splncs04}
\bibliography{rescience}
%
\end{document}
